% Options for packages loaded elsewhere
\PassOptionsToPackage{unicode}{hyperref}
\PassOptionsToPackage{hyphens}{url}
\PassOptionsToPackage{dvipsnames,svgnames,x11names}{xcolor}
%
\documentclass[
  12pt,
]{article}
\usepackage{amsmath,amssymb}
\usepackage{iftex}
\ifPDFTeX
  \usepackage[T1]{fontenc}
  \usepackage[utf8]{inputenc}
  \usepackage{textcomp} % provide euro and other symbols
\else % if luatex or xetex
  \usepackage{unicode-math} % this also loads fontspec
  \defaultfontfeatures{Scale=MatchLowercase}
  \defaultfontfeatures[\rmfamily]{Ligatures=TeX,Scale=1}
\fi
\usepackage{lmodern}
\ifPDFTeX\else
  % xetex/luatex font selection
  \setmainfont[]{Palatino}
  \setsansfont[]{Helvetica}
  \setmonofont[]{Menlo}
\fi
% Use upquote if available, for straight quotes in verbatim environments
\IfFileExists{upquote.sty}{\usepackage{upquote}}{}
\IfFileExists{microtype.sty}{% use microtype if available
  \usepackage[]{microtype}
  \UseMicrotypeSet[protrusion]{basicmath} % disable protrusion for tt fonts
}{}
\makeatletter
\@ifundefined{KOMAClassName}{% if non-KOMA class
  \IfFileExists{parskip.sty}{%
    \usepackage{parskip}
  }{% else
    \setlength{\parindent}{0pt}
    \setlength{\parskip}{6pt plus 2pt minus 1pt}}
}{% if KOMA class
  \KOMAoptions{parskip=half}}
\makeatother
\usepackage{xcolor}
\usepackage[margin = 1.2in]{geometry}
\usepackage{color}
\usepackage{fancyvrb}
\newcommand{\VerbBar}{|}
\newcommand{\VERB}{\Verb[commandchars=\\\{\}]}
\DefineVerbatimEnvironment{Highlighting}{Verbatim}{commandchars=\\\{\}}
% Add ',fontsize=\small' for more characters per line
\newenvironment{Shaded}{}{}
\newcommand{\AlertTok}[1]{\textcolor[rgb]{1.00,0.00,0.00}{\textbf{#1}}}
\newcommand{\AnnotationTok}[1]{\textcolor[rgb]{0.38,0.63,0.69}{\textbf{\textit{#1}}}}
\newcommand{\AttributeTok}[1]{\textcolor[rgb]{0.49,0.56,0.16}{#1}}
\newcommand{\BaseNTok}[1]{\textcolor[rgb]{0.25,0.63,0.44}{#1}}
\newcommand{\BuiltInTok}[1]{\textcolor[rgb]{0.00,0.50,0.00}{#1}}
\newcommand{\CharTok}[1]{\textcolor[rgb]{0.25,0.44,0.63}{#1}}
\newcommand{\CommentTok}[1]{\textcolor[rgb]{0.38,0.63,0.69}{\textit{#1}}}
\newcommand{\CommentVarTok}[1]{\textcolor[rgb]{0.38,0.63,0.69}{\textbf{\textit{#1}}}}
\newcommand{\ConstantTok}[1]{\textcolor[rgb]{0.53,0.00,0.00}{#1}}
\newcommand{\ControlFlowTok}[1]{\textcolor[rgb]{0.00,0.44,0.13}{\textbf{#1}}}
\newcommand{\DataTypeTok}[1]{\textcolor[rgb]{0.56,0.13,0.00}{#1}}
\newcommand{\DecValTok}[1]{\textcolor[rgb]{0.25,0.63,0.44}{#1}}
\newcommand{\DocumentationTok}[1]{\textcolor[rgb]{0.73,0.13,0.13}{\textit{#1}}}
\newcommand{\ErrorTok}[1]{\textcolor[rgb]{1.00,0.00,0.00}{\textbf{#1}}}
\newcommand{\ExtensionTok}[1]{#1}
\newcommand{\FloatTok}[1]{\textcolor[rgb]{0.25,0.63,0.44}{#1}}
\newcommand{\FunctionTok}[1]{\textcolor[rgb]{0.02,0.16,0.49}{#1}}
\newcommand{\ImportTok}[1]{\textcolor[rgb]{0.00,0.50,0.00}{\textbf{#1}}}
\newcommand{\InformationTok}[1]{\textcolor[rgb]{0.38,0.63,0.69}{\textbf{\textit{#1}}}}
\newcommand{\KeywordTok}[1]{\textcolor[rgb]{0.00,0.44,0.13}{\textbf{#1}}}
\newcommand{\NormalTok}[1]{#1}
\newcommand{\OperatorTok}[1]{\textcolor[rgb]{0.40,0.40,0.40}{#1}}
\newcommand{\OtherTok}[1]{\textcolor[rgb]{0.00,0.44,0.13}{#1}}
\newcommand{\PreprocessorTok}[1]{\textcolor[rgb]{0.74,0.48,0.00}{#1}}
\newcommand{\RegionMarkerTok}[1]{#1}
\newcommand{\SpecialCharTok}[1]{\textcolor[rgb]{0.25,0.44,0.63}{#1}}
\newcommand{\SpecialStringTok}[1]{\textcolor[rgb]{0.73,0.40,0.53}{#1}}
\newcommand{\StringTok}[1]{\textcolor[rgb]{0.25,0.44,0.63}{#1}}
\newcommand{\VariableTok}[1]{\textcolor[rgb]{0.10,0.09,0.49}{#1}}
\newcommand{\VerbatimStringTok}[1]{\textcolor[rgb]{0.25,0.44,0.63}{#1}}
\newcommand{\WarningTok}[1]{\textcolor[rgb]{0.38,0.63,0.69}{\textbf{\textit{#1}}}}
\setlength{\emergencystretch}{3em} % prevent overfull lines
\providecommand{\tightlist}{%
  \setlength{\itemsep}{0pt}\setlength{\parskip}{0pt}}
\setcounter{secnumdepth}{5}
\ifLuaTeX
\usepackage[bidi=basic]{babel}
\else
\usepackage[bidi=default]{babel}
\fi
\babelprovide[main,import]{french}
\ifPDFTeX
\else
\babelfont{rm}[]{Palatino}
\fi
% get rid of language-specific shorthands (see #6817):
\let\LanguageShortHands\languageshorthands
\def\languageshorthands#1{}
\ifLuaTeX
  \usepackage{selnolig}  % disable illegal ligatures
\fi
\IfFileExists{bookmark.sty}{\usepackage{bookmark}}{\usepackage{hyperref}}
\IfFileExists{xurl.sty}{\usepackage{xurl}}{} % add URL line breaks if available
\urlstyle{same}
\hypersetup{
  pdftitle={Rapport Projet TAS - Implémentation d'un langage de programmation avec typage},
  pdfauthor={Dibassi Brahima 21210230},
  pdflang={fr},
  colorlinks=true,
  linkcolor={Maroon},
  filecolor={Maroon},
  citecolor={Blue},
  urlcolor={NavyBlue},
  pdfcreator={LaTeX via pandoc}}

\title{Rapport Projet TAS - Implémentation d'un langage de programmation
avec typage}
\usepackage{etoolbox}
\makeatletter
\providecommand{\subtitle}[1]{% add subtitle to \maketitle
  \apptocmd{\@title}{\par {\large #1 \par}}{}{}
}
\makeatother
\subtitle{Typeur, Evaluateur , Parser}
\author{Dibassi Brahima 21210230}
\date{2023-11-19}

\begin{document}
\maketitle

\newpage

{
\hypersetup{linkcolor=}
\setcounter{tocdepth}{3}
\tableofcontents
}
\newpage

\section{Langage de programmation
choisi}\label{langage-de-programmation-choisi}

Afin d'implémenter ce projet, nous avons choisi le langage de
programmation \emph{OCaml}.\\
Ayant déjà expérimenté l'implémentation de l'interpréteur d'un langage
en OCaml suite au cours d' \textbf{APS}, nous avons pu constater que ce
langage était très adapté à ce genre de projet.\\
Grâce à la somme de produits au pattern-matching associé, et au support
des paradigmes, imperatif et fonctionnel, nous avons pu nous servir des
différents traits du langage afin de faciliter notre implémentation de
l'évaluateur et du typeur.

\section{Lancement du projet}\label{lancement-du-projet}

Afin de compiler notre projet il suffit d'être à la racine de chaque
version d'APS et de lancer :

\begin{Shaded}
\begin{Highlighting}[]
    \ExtensionTok{dune}\NormalTok{ build}
\end{Highlighting}
\end{Shaded}

Pour les versions du langage supérieures à APS1A, nous proposons
également un lancement automatisé de notre batterie de tests. La
commande est :

\begin{Shaded}
\begin{Highlighting}[]
    \ExtensionTok{dune}\NormalTok{ test }\AttributeTok{{-}{-}force}
\end{Highlighting}
\end{Shaded}

Si vous souhaitez lancer votre fichier à travers toutes les étapes, il
est également possible d'utiliser la commande suivante :

\begin{Shaded}
\begin{Highlighting}[]
    \ExtensionTok{dune}\NormalTok{ exec ProjetTAS }\PreprocessorTok{*}\NormalTok{chemin vers le fichier à tester}\PreprocessorTok{*}\NormalTok{/}\PreprocessorTok{*}\NormalTok{fichier}\PreprocessorTok{*}\NormalTok{.ml}
\end{Highlighting}
\end{Shaded}

\newpage

\section{Structure du projet}\label{structure-du-projet}

\begin{itemize}
\tightlist
\item
  Repertoire \texttt{Lib/}

  \begin{itemize}
  \tightlist
  \item
    \texttt{lexer.mll} : Création des tokens
  \item
    \texttt{parser.mly} : Parser afin de construire notre arbre
    syntaxique abstrait depuis les tokens
  \item
    \texttt{ast.ml} : Structure de l'AST du langage avec les types des
    expressions
  \item
    \texttt{baselib.ml} : Définitions des primitives du langage et leur
    types
  \item
    \texttt{evaluator.ml} : Partie sémentique du langage, évaluation des
    expressions
  \item
    \texttt{prettyprinter.ml} : Fonctions d'affichage
  \item
    \texttt{typeur.ml} : Partie analyse statique, vérification des types
  \item
    \texttt{typingEnv.ml} : Gestion de l'environnement des déclarations
    de types par l'utilisateur
  \end{itemize}
\item
  Repertoire \texttt{Bin/}

  \begin{itemize}
  \tightlist
  \item
    \texttt{main.ml} : Contient notre script principal, lancé lors de la
    commande \texttt{dune\ exec\ ProjetTAS\ *fichier*.ml}. Il vérifie si
    le programme .ml ciblé est bien typé et l'évalue si c'est le cas en
    faisant tous les affichages sur le terminal. Nous avons choisi
    d'afficher également l'état de la mémoire à la fin de l'évaluation
    du programme.
  \end{itemize}
\item
  Repertoire \texttt{Test/}

  \begin{itemize}
  \tightlist
  \item
    \texttt{testing.ml} : Script de test, permet de tester tous les
    fichiers .ml présents dans le dossier \texttt{test} et de vérifier
    si le typeur et l'évaluateur fonctionnent correctement.
  \item
    \texttt{yamlHelper.ml} : Fonctions d'aide pour la lecture des
    fichiers de test au format yaml
  \item
    \texttt{template.yaml} : Template de fichier de test au format yaml
  \end{itemize}
\end{itemize}

Les tests sont dans le dossier \texttt{test} classés en fonction des
extentions du langage auxquels ils correspondent. Les extensions du
langage devant également passer les tests des versions précédentes.

\newpage

\section{Syntaxe du langage}\label{syntaxe-du-langage}

\section{Difficultés rencontrées}\label{difficultuxe9s-rencontruxe9es}

\section{Extensions}\label{extensions}

\subsection{Types Utilisateurs}\label{types-utilisateurs}

\subsection{MatchPattern}\label{matchpattern}

\subsection{Annotations de type}\label{annotations-de-type}

\subsection{Gestion des erreurs}\label{gestion-des-erreurs}

\end{document}
