% Options for packages loaded elsewhere
\PassOptionsToPackage{unicode}{hyperref}
\PassOptionsToPackage{hyphens}{url}
\PassOptionsToPackage{dvipsnames,svgnames,x11names}{xcolor}
%
\documentclass[
  12pt,
]{article}
\usepackage{amsmath,amssymb}
\usepackage{iftex}
\ifPDFTeX
  \usepackage[T1]{fontenc}
  \usepackage[utf8]{inputenc}
  \usepackage{textcomp} % provide euro and other symbols
\else % if luatex or xetex
  \usepackage{unicode-math} % this also loads fontspec
  \defaultfontfeatures{Scale=MatchLowercase}
  \defaultfontfeatures[\rmfamily]{Ligatures=TeX,Scale=1}
\fi
\usepackage{lmodern}
\ifPDFTeX\else
  % xetex/luatex font selection
  \setmainfont[]{Palatino}
  \setsansfont[]{Helvetica}
  \setmonofont[]{Menlo}
\fi
% Use upquote if available, for straight quotes in verbatim environments
\IfFileExists{upquote.sty}{\usepackage{upquote}}{}
\IfFileExists{microtype.sty}{% use microtype if available
  \usepackage[]{microtype}
  \UseMicrotypeSet[protrusion]{basicmath} % disable protrusion for tt fonts
}{}
\makeatletter
\@ifundefined{KOMAClassName}{% if non-KOMA class
  \IfFileExists{parskip.sty}{%
    \usepackage{parskip}
  }{% else
    \setlength{\parindent}{0pt}
    \setlength{\parskip}{6pt plus 2pt minus 1pt}}
}{% if KOMA class
  \KOMAoptions{parskip=half}}
\makeatother
\usepackage{xcolor}
\usepackage[margin = 1.2in]{geometry}
\usepackage{color}
\usepackage{fancyvrb}
\newcommand{\VerbBar}{|}
\newcommand{\VERB}{\Verb[commandchars=\\\{\}]}
\DefineVerbatimEnvironment{Highlighting}{Verbatim}{commandchars=\\\{\}}
% Add ',fontsize=\small' for more characters per line
\newenvironment{Shaded}{}{}
\newcommand{\AlertTok}[1]{\textcolor[rgb]{1.00,0.00,0.00}{\textbf{#1}}}
\newcommand{\AnnotationTok}[1]{\textcolor[rgb]{0.38,0.63,0.69}{\textbf{\textit{#1}}}}
\newcommand{\AttributeTok}[1]{\textcolor[rgb]{0.49,0.56,0.16}{#1}}
\newcommand{\BaseNTok}[1]{\textcolor[rgb]{0.25,0.63,0.44}{#1}}
\newcommand{\BuiltInTok}[1]{\textcolor[rgb]{0.00,0.50,0.00}{#1}}
\newcommand{\CharTok}[1]{\textcolor[rgb]{0.25,0.44,0.63}{#1}}
\newcommand{\CommentTok}[1]{\textcolor[rgb]{0.38,0.63,0.69}{\textit{#1}}}
\newcommand{\CommentVarTok}[1]{\textcolor[rgb]{0.38,0.63,0.69}{\textbf{\textit{#1}}}}
\newcommand{\ConstantTok}[1]{\textcolor[rgb]{0.53,0.00,0.00}{#1}}
\newcommand{\ControlFlowTok}[1]{\textcolor[rgb]{0.00,0.44,0.13}{\textbf{#1}}}
\newcommand{\DataTypeTok}[1]{\textcolor[rgb]{0.56,0.13,0.00}{#1}}
\newcommand{\DecValTok}[1]{\textcolor[rgb]{0.25,0.63,0.44}{#1}}
\newcommand{\DocumentationTok}[1]{\textcolor[rgb]{0.73,0.13,0.13}{\textit{#1}}}
\newcommand{\ErrorTok}[1]{\textcolor[rgb]{1.00,0.00,0.00}{\textbf{#1}}}
\newcommand{\ExtensionTok}[1]{#1}
\newcommand{\FloatTok}[1]{\textcolor[rgb]{0.25,0.63,0.44}{#1}}
\newcommand{\FunctionTok}[1]{\textcolor[rgb]{0.02,0.16,0.49}{#1}}
\newcommand{\ImportTok}[1]{\textcolor[rgb]{0.00,0.50,0.00}{\textbf{#1}}}
\newcommand{\InformationTok}[1]{\textcolor[rgb]{0.38,0.63,0.69}{\textbf{\textit{#1}}}}
\newcommand{\KeywordTok}[1]{\textcolor[rgb]{0.00,0.44,0.13}{\textbf{#1}}}
\newcommand{\NormalTok}[1]{#1}
\newcommand{\OperatorTok}[1]{\textcolor[rgb]{0.40,0.40,0.40}{#1}}
\newcommand{\OtherTok}[1]{\textcolor[rgb]{0.00,0.44,0.13}{#1}}
\newcommand{\PreprocessorTok}[1]{\textcolor[rgb]{0.74,0.48,0.00}{#1}}
\newcommand{\RegionMarkerTok}[1]{#1}
\newcommand{\SpecialCharTok}[1]{\textcolor[rgb]{0.25,0.44,0.63}{#1}}
\newcommand{\SpecialStringTok}[1]{\textcolor[rgb]{0.73,0.40,0.53}{#1}}
\newcommand{\StringTok}[1]{\textcolor[rgb]{0.25,0.44,0.63}{#1}}
\newcommand{\VariableTok}[1]{\textcolor[rgb]{0.10,0.09,0.49}{#1}}
\newcommand{\VerbatimStringTok}[1]{\textcolor[rgb]{0.25,0.44,0.63}{#1}}
\newcommand{\WarningTok}[1]{\textcolor[rgb]{0.38,0.63,0.69}{\textbf{\textit{#1}}}}
\setlength{\emergencystretch}{3em} % prevent overfull lines
\providecommand{\tightlist}{%
  \setlength{\itemsep}{0pt}\setlength{\parskip}{0pt}}
\setcounter{secnumdepth}{5}
\ifLuaTeX
\usepackage[bidi=basic]{babel}
\else
\usepackage[bidi=default]{babel}
\fi
\babelprovide[main,import]{french}
\ifPDFTeX
\else
\babelfont{rm}[]{Palatino}
\fi
% get rid of language-specific shorthands (see #6817):
\let\LanguageShortHands\languageshorthands
\def\languageshorthands#1{}
\ifLuaTeX
  \usepackage{selnolig}  % disable illegal ligatures
\fi
\IfFileExists{bookmark.sty}{\usepackage{bookmark}}{\usepackage{hyperref}}
\IfFileExists{xurl.sty}{\usepackage{xurl}}{} % add URL line breaks if available
\urlstyle{same}
\hypersetup{
  pdftitle={Rapport Projet TAS - Implémentation d'un langage de programmation avec typage},
  pdfauthor={Dibassi Brahima 21210230},
  pdflang={fr},
  colorlinks=true,
  linkcolor={Maroon},
  filecolor={Maroon},
  citecolor={Blue},
  urlcolor={NavyBlue},
  pdfcreator={LaTeX via pandoc}}

\title{Rapport Projet TAS - Implémentation d'un langage de programmation
avec typage}
\usepackage{etoolbox}
\makeatletter
\providecommand{\subtitle}[1]{% add subtitle to \maketitle
  \apptocmd{\@title}{\par {\large #1 \par}}{}{}
}
\makeatother
\subtitle{Typeur, Evaluateur , Analyseur Syntaxique}
\author{Dibassi Brahima 21210230}
\date{2023-11-19}

\begin{document}
\maketitle

\newpage

{
\hypersetup{linkcolor=}
\setcounter{tocdepth}{3}
\tableofcontents
}
\newpage

\newcommand{\grammarRule}[1]{\; \textbf{<\textcolor{Blue}{#1}>} \;}
\newcommand{\grammarRuleUnSpaced}[1]{\textbf{<\textcolor{Blue}{#1}>}}
\newcommand{\nTime}[1]{\; #1\text{*} \;}
\newcommand{\nPlus}[1]{\; #1\text{+} \;}
\newcommand{\isToken}[1]{\; \textit{`\textcolor{Maroon}{#1}`} \;}
\newcommand{\isRangeToken}[2]{\; \textit{`\textcolor{Maroon}{#1} - \textcolor{Maroon}{#2}`} \;}

\section{Langage de programmation
choisi}\label{langage-de-programmation-choisi}

Afin d'implémenter ce projet, nous avons choisi le langage de
programmation \emph{OCaml}.\\
Ayant déjà expérimenté l'implémentation de l'interpréteur d'un langage
en OCaml suite au cours d' \textbf{APS}, nous avons pu constater que ce
langage était très adapté à ce genre de projet.\\
Grâce à la somme de produits au pattern-matching associé, et au support
des paradigmes, impératif et fonctionnel, nous avons pu nous servir des
différents traits du langage afin de faciliter notre implémentation de
l'évaluateur et du typeur.

\section{Lancement du projet}\label{lancement-du-projet}

Si vous souhaitez lancer votre fichier utilisez la commande suivante :

\begin{Shaded}
\begin{Highlighting}[]
    \ExtensionTok{dune}\NormalTok{ exec ProjetTAS }\PreprocessorTok{*}\NormalTok{chemin vers le fichier à tester}\PreprocessorTok{*}\NormalTok{/}\PreprocessorTok{*}\NormalTok{fichier}\PreprocessorTok{*}\NormalTok{.ml}
\end{Highlighting}
\end{Shaded}

Nous proposons également un lancement automatisé de notre batterie de
tests. La commande est :

\begin{Shaded}
\begin{Highlighting}[]
    \ExtensionTok{dune}\NormalTok{ test }\AttributeTok{{-}{-}force}
\end{Highlighting}
\end{Shaded}

\newpage

\section{Structure du projet}\label{structure-du-projet}

\begin{itemize}
\tightlist
\item
  Repertoire \texttt{Lib/}

  \begin{itemize}
  \tightlist
  \item
    \texttt{lexer.mll} : Création des tokens
  \item
    \texttt{parser.mly} : Parser afin de construire notre arbre
    syntaxique abstrait depuis les tokens
  \item
    \texttt{ast.ml} : Structure de l'AST du langage avec les types des
    expressions
  \item
    \texttt{baselib.ml} : Définitions des primitives du langage et leur
    types
  \item
    \texttt{evaluator.ml} : Partie sémentique du langage, évaluation des
    expressions
  \item
    \texttt{prettyprinter.ml} : Fonctions d'affichage
  \item
    \texttt{typeur.ml} : Partie analyse statique, vérification des types
  \item
    \texttt{typingEnv.ml} : Gestion de l'environnement des déclarations
    de types par l'utilisateur
  \end{itemize}
\item
  Repertoire \texttt{Bin/}

  \begin{itemize}
  \tightlist
  \item
    \texttt{main.ml} : Contient notre script principal, lancé lors de la
    commande \texttt{dune\ exec\ ProjetTAS\ *fichier*.ml}. Il vérifie si
    le programme .ml ciblé est bien typé et l'évalue si c'est le cas en
    faisant tous les affichages sur le terminal. Nous avons choisi
    d'afficher également l'état de la mémoire à la fin de l'évaluation
    du programme.
  \end{itemize}
\item
  Repertoire \texttt{Test/}

  \begin{itemize}
  \tightlist
  \item
    \texttt{testing.ml} : Script de test, permet de tester tous les
    fichiers .ml présents dans le dossier \texttt{test} et de vérifier
    si le typeur et l'évaluateur fonctionnent correctement.
  \item
    \texttt{yamlHelper.ml} : Fonctions d'aide pour la lecture des
    fichiers de test au format yaml
  \item
    \texttt{template.yaml} : Template de fichier de test au format yaml
  \end{itemize}
\end{itemize}

Les tests sont dans le dossier \texttt{test} classés en fonction des
extentions du langage auxquels ils correspondent. Les extensions du
langage devant également passer les tests des versions précédentes.

\newpage

\section{Syntaxe du langage}\label{syntaxe-du-langage}

Une grande partie de la syntaxe du langage est reprise d'un projet
précédent, le langage MiniML.\\

\subsection{Programme}\label{programme}

On définit un programme comme une expression précédée de zéro ou
plusieurs définitions néccessaires à son évaluation. \begin{align*}
      \; \textbf{<\textcolor{Blue}{Prog}>} \; ::= \quad & |\quad \; \textbf{<\textcolor{Blue}{Expr}>} \;                                  \\
                                   & |\quad \; \textbf{<\textcolor{Blue}{Def}>} \;  \; \textit{`\textcolor{Maroon}{;;}`} \;  \; \textbf{<\textcolor{Blue}{Prog}>} \;
\end{align*}

\subsection{Elément nommés}\label{eluxe9ment-nommuxe9s}

On appelle éléments nommés, les identificateurs
\; \textbf{<\textcolor{Blue}{Id}>} \; pour les variables, motifs et
types, les identificateurs de constructeurs
\; \textbf{<\textcolor{Blue}{ConstructeurId}>} \; et les variables de
types \; \textbf{<\textcolor{Blue}{Vartype}>} \;

\begin{align*}
      \; \textbf{<\textcolor{Blue}{Id}>} \; ::= \quad             & \; [\; \textit{`\textcolor{Maroon}{a} - \textcolor{Maroon}{z}`} \; \; \textit{`\textcolor{Maroon}{A} - \textcolor{Maroon}{Z}`} \; \; \textit{`\textcolor{Maroon}{0} - \textcolor{Maroon}{9}`} \; \; \textit{`\textcolor{Maroon}{\_}`} \;]\text{+} \; \\
      \; \textbf{<\textcolor{Blue}{ConstructeurId}>} \; ::= \quad & [\; \textit{`\textcolor{Maroon}{A} - \textcolor{Maroon}{Z}`} \;] \; \textbf{<\textcolor{Blue}{Id}>} \;                                             \\
      \; \textbf{<\textcolor{Blue}{Vartype}>} \; ::= \quad        & \; \textit{`\textcolor{Maroon}{'}`} \;[\; \textit{`\textcolor{Maroon}{a} - \textcolor{Maroon}{z}`} \;] \; \; [\; \textit{`\textcolor{Maroon}{0} - \textcolor{Maroon}{9}`} \;]\text{*} \;
\end{align*}

\subsection{Types}\label{types}

\begin{itemize}
      \tightlist
      \item
            Les \textbf{Types polymorphique}.
      \item
            Les \textbf{Utilisation de variables de types}.
      \item
            Les \textbf{Applications de types}.
      \item
            Les \textbf{Lambda}.
      \item
            Les \textbf{Tuples}.
\end{itemize}

\begin{align*}
      \; \textbf{<\textcolor{Blue}{Type}>} \;    \quad ::=  \quad & |\quad \; \textbf{<\textcolor{Blue}{Vartype}>} \;                              \\
                                             & |\quad \; \textbf{<\textcolor{Blue}{Id}>} \;                                   \\
                                             & |\quad \; \textbf{<\textcolor{Blue}{Type}>} \;   \; \textbf{<\textcolor{Blue}{Type}>} \;            \\
                                             & |\quad \; \textbf{<\textcolor{Blue}{Type}>} \; \; \textit{`\textcolor{Maroon}{->}`} \; \; \textbf{<\textcolor{Blue}{Type}>} \; \\
                                             & |\quad \; \textbf{<\textcolor{Blue}{Type}>} \; \; \textit{`\textcolor{Maroon}{*}`} \;  \; \textbf{<\textcolor{Blue}{Type}>} \; \\
\end{align*}

\pagebreak

\subsection{Déclarations de types}\label{duxe9clarations-de-types}

\begin{align*}
                                  & \quad \; \textit{`\textcolor{Maroon}{type}`} \; \; \textbf{<\textcolor{Blue}{Vartype}>}\text{*} \; \; \textbf{<\textcolor{Blue}{Id}>} \; \; \textit{`\textcolor{Maroon}{=}`} \; \; \textbf{<\textcolor{Blue}{NewContructors}>} \; \\
\end{align*}

\subsubsection{Nouveaux constructeurs}\label{nouveaux-constructeurs}

\begin{align*}
      \; \textbf{<\textcolor{Blue}{NewContructors}>} \; ::=   \quad & |\quad  \; \textbf{<\textcolor{Blue}{ConstructeurId}>} \; \; \textit{`\textcolor{Maroon}{of}`} \; \; \textbf{<\textcolor{Blue}{Type}>} \;                                            \\
                                               & |\quad  \; \textbf{<\textcolor{Blue}{NewContructors}>} \; \; \textit{`\textcolor{Maroon}{|}`} \; \; \textbf{<\textcolor{Blue}{NewContructors}>} \;                                   \\
\end{align*}

\subsection{Expressions}\label{expressions}

\begin{itemize}
      \tightlist
      \item
            Les \textbf{Littéraux}.
      \item
            Les \textbf{Utilisations de Variables}.
      \item
            Les \textbf{Appels d'opérateurs} unaire et binaire.
      \item
            Les \textbf{Appels de fonctions}.
      \item
            Les \textbf{Tuples}.
      \item
            Les \textbf{Lambda}.
      \item
            Les \textbf{Fonctions Récursives}.
      \item
            Les \textbf{Constructions}.
      \item
            Les \textbf{Correspondance de motifs}.
\end{itemize}

\begin{align*}
      \; \textbf{<\textcolor{Blue}{Litteral}>} \;  \quad ::=  \quad       & |\quad \; [\; \textit{`\textcolor{Maroon}{0} - \textcolor{Maroon}{9}`} \;]\text{+} \;                                                                                                                                                  \\
                                                     & |\quad \; \textit{`\textcolor{Maroon}{(}`} \; \; \textit{`\textcolor{Maroon}{)}`} \;                                                                                                                                                        \\
      \\
      \; \textbf{<\textcolor{Blue}{Expr}>} \;  \quad ::=  \quad           & |\quad \; \textbf{<\textcolor{Blue}{Litteral}>} \;                                                                                                                                                         \\
                                                     & |\quad \; \textbf{<\textcolor{Blue}{Id}>} \;                                                                                                                                                               \\
                                                     & |\quad \; \textbf{<\textcolor{Blue}{UnaryOperator}>} \;  \; \textbf{<\textcolor{Blue}{Expr}>} \;                                                                                                                                \\
                                                     & |\quad  \; \textbf{<\textcolor{Blue}{Expr}>} \; \; \textbf{<\textcolor{Blue}{BinaryOperator}>} \;    \; \textbf{<\textcolor{Blue}{Expr}>} \;                                                                                                         \\
                                                     & |\quad \; \textbf{<\textcolor{Blue}{Expr}>} \;  \; \textbf{<\textcolor{Blue}{Expr}>} \;                                                                                                                                         \\
                                                     & |\quad \; \textbf{<\textcolor{Blue}{Expr}>} \; \; \textit{`\textcolor{Maroon}{,}`} \;  \; \textbf{<\textcolor{Blue}{Expr}>} \;                                                                                                                             \\
                                                     & |\quad \; \textit{`\textcolor{Maroon}{fun}`} \; \; \textbf{<\textcolor{Blue}{Id}>} \; \; \textit{`\textcolor{Maroon}{->}`} \;  \; \textbf{<\textcolor{Blue}{Expr}>} \;                                                                                                                \\
                                                     & |\quad \; \textit{`\textcolor{Maroon}{let}`} \; \; \textbf{<\textcolor{Blue}{Id}>} \; \; \textbf{<\textcolor{Blue}{Id}>} \; \; \textit{`\textcolor{Maroon}{=}`} \;  \; \textbf{<\textcolor{Blue}{Expr}>} \;  \; \textit{`\textcolor{Maroon}{in}`} \;  \; \textbf{<\textcolor{Blue}{Expr}>} \;                                                                                                         \\
                        & |\quad \; \textit{`\textcolor{Maroon}{let}`} \; \; \textit{`\textcolor{Maroon}{rec}`} \;  \; \textbf{<\textcolor{Blue}{Id}>} \; \; \textit{`\textcolor{Maroon}{=}`} \;  \; \textbf{<\textcolor{Blue}{Expr}>} \;  \; \textit{`\textcolor{Maroon}{in}`} \;  \; \textbf{<\textcolor{Blue}{Expr}>} \;                                         
                                                                                        \\
                                                     & |\quad \; \textbf{<\textcolor{Blue}{ConstructeurId}>} \;  \; \textbf{<\textcolor{Blue}{Expr}>} \;                                                                                                                               \\
                                                     & |\quad \; \textit{`\textcolor{Maroon}{match}`} \; \; \textbf{<\textcolor{Blue}{Expr}>} \; \; \textit{`\textcolor{Maroon}{with}`} \; \; \textbf{<\textcolor{Blue}{MatchCase}>} \;                                                                                                      \\
\end{align*}

\subsection{Filtrage de motifs}\label{filtrage-de-motifs}

\begin{itemize}
      \tightlist
      \item
            Les patterns sur \textbf{Littéraux}.
      \item
            Les patterns sur \textbf{Variables}.
      \item
            Les patterns sur \textbf{Tuple}.
      \item
            Les patterns sur \textbf{Constructeurs}.
\end{itemize}

\begin{align*}
      \; \textbf{<\textcolor{Blue}{MatchCase}>} \;  \quad ::=  \quad & |\quad  \; \textbf{<\textcolor{Blue}{Pattern}>} \; \; \textit{`\textcolor{Maroon}{->}`} \;  \; \textbf{<\textcolor{Blue}{Expr}>} \;     \\
                                                & |\quad \; \textbf{<\textcolor{Blue}{MatchCase}>} \; \; \textit{`\textcolor{Maroon}{|}`} \; \; \textbf{<\textcolor{Blue}{MatchCase}>} \; \\
      \\
      \; \textbf{<\textcolor{Blue}{Pattern}>} \; \quad ::=  \quad    & |\quad \; \textbf{<\textcolor{Blue}{Litteral}>} \;                                      \\
                                                & |\quad \; \textbf{<\textcolor{Blue}{Id}>} \;                                            \\
                                                & |\quad \; \textbf{<\textcolor{Blue}{Pattern}>} \;  \; \textit{`\textcolor{Maroon}{,}`} \; \; \textbf{<\textcolor{Blue}{Pattern}>} \;    \\
                                                & |\quad \; \textbf{<\textcolor{Blue}{ConstructeurId}>} \; \; \; \textbf{<\textcolor{Blue}{Pattern}>} \;       \\
\end{align*}

\pagebreak

\section{Let Polymorphique}\label{let-polymorphique}

Durant la réalisation de ce projet, nous avons rencontré une difficulté
majeure, celle de la gestion du polymorphisme qui a été un véritable
challenge pour nous.\\

En effet, nous avons dû faire face à plusieurs problèmes :

\begin{enumerate}
      \item
            La gestion de l'environnement des types
      \item
            La généralisation des types
      \item
            La remontée des substitutions induite par l'unification
\end{enumerate}

Afin de résoudre ces problèmes, nous avons dû revoir notre gestion de
l'environnement des types plusieurs fois.\\

Dans un premier temps, notre environnement des types était un simple set
de variables de types. Collectant les variables de types lors de leur
création. Hélas cette implémentation ne nous permettait pas de récupérer
les nouvelles variables de type créées lors de la récupération des
substitutions.\\

La solution que nous avons mise en place afin de résoudre ce problème a
été de récupérer toutes les substitutions induites par le typeur et de
les appliquer à l'environnement des types. De plus, nous ajoutons toutes
les substitutions induites par la généralisation à la génération
d'équations courante.

\pagebreak

\section{Extensions}\label{extensions}

Dans cette partie nous allons présenter les différentes extensions que
nous avons implémentées dans notre langage. Pour chacune d'entre elles
une implémentation de l'évaluateur et du typeur a été réalisée.

\subsection{Annotations de type}\label{annotations-de-type}

Afin de simplifier les tests sur notre typeur, nous avons implémenté la
possibilité d'annoter les expressions avec leur type attendu.\\

\begin{Shaded}
\begin{Highlighting}[]
\KeywordTok{let}\NormalTok{ intId a = (a : }\DataTypeTok{int}\NormalTok{) }\KeywordTok{in}\NormalTok{ (intId }\DecValTok{1}\NormalTok{) }\CommentTok{(* Type *)}
\KeywordTok{let}\NormalTok{ intId a = (a : }\DataTypeTok{int}\NormalTok{) }\KeywordTok{in}\NormalTok{ (intId (}\KeywordTok{fun}\NormalTok{ a {-}\textgreater{} a)) }\CommentTok{(* Type Error *)}
\end{Highlighting}
\end{Shaded}

\subsection{Types Utilisateurs}\label{types-utilisateurs}

Pour generaliser les cas listes chainées et option, nous avons
implémenté la possibilité de définir des types utilisateurs.

\begin{Shaded}
\begin{Highlighting}[]
\KeywordTok{type}\NormalTok{ \textquotesingle{}a }\DataTypeTok{list}\NormalTok{ = }
\NormalTok{  | Nil }
\NormalTok{  | Cons }\KeywordTok{of}\NormalTok{ (\textquotesingle{}a * (\textquotesingle{}a }\DataTypeTok{list}\NormalTok{))}

\KeywordTok{type}\NormalTok{ \textquotesingle{}a }\DataTypeTok{option}\NormalTok{ = }
\NormalTok{  | }\DataTypeTok{None}
\NormalTok{  | }\DataTypeTok{Some} \KeywordTok{of}\NormalTok{ \textquotesingle{}a}
\end{Highlighting}
\end{Shaded}

Au sein du typeur, ces définitions de types sont traduites en source
d'instances de type. Ainsi, lors de la vérification de type, les types
utilisateurs sont traités comme des types issus d'une généralisation.
Pour pouvoir utiliser ces définitions, un environnement est créé au
début du programme afin de mettre en association les constructeurs avec
leur définitions et le type qu'ils construisent.

\pagebreak

\subsection{MatchPattern Profond}\label{matchpattern-profond}

Afin de pouvoir utiliser correctement les types utilisateurs, nous avons
implémenté la possibilité de faire du filtrage de motifs profonds.\\

Il prend la forme suivante :

\begin{Shaded}
\begin{Highlighting}[]
\KeywordTok{let}\NormalTok{ t = }\DataTypeTok{Some}\NormalTok{ (}\DataTypeTok{Some}\NormalTok{ (}\DecValTok{1}\NormalTok{)) }\KeywordTok{in} 
\KeywordTok{match}\NormalTok{ t }\KeywordTok{with}
\NormalTok{| }\DataTypeTok{Some}\NormalTok{ (}\DataTypeTok{Some}\NormalTok{ x) {-}\textgreater{} x}
\NormalTok{| }\DataTypeTok{Some}\NormalTok{ (}\DataTypeTok{Some}\NormalTok{ x) {-}\textgreater{} (x : (}\DataTypeTok{int} \DataTypeTok{ref}\NormalTok{)) }\CommentTok{(* Ici ne type pas *)}
\NormalTok{| }\DataTypeTok{None}\NormalTok{ {-}\textgreater{} }\DecValTok{0}
\end{Highlighting}
\end{Shaded}

\subsubsection{Analyse Statique}\label{analyse-statique}

L' analyse Statique vérifie ici 4 informations :

\begin{enumerate}
      \item
            Le type des patterns est bien conforme au type de l'expression filtrée
      \item
            Le type de retour de chaque branche est bien conforme au type de retour de l'expression filtrée
      \item
            Que l'expression de retour de chaque branche n'utilise pas de variables non définies
      \item
            L'absence de doublons dans les variables de motifs
\end{enumerate}

\subsubsection{Evaluation}\label{evaluation}

L'évaluation s'est avérée un peu plus complexe, car il a fallu mettre en
place un parcours de graphe afin de pouvoir associer les variables de
motifs aux valeurs correspondantes dans l'expression filtrée.

\pagebreak

\subsection{Gestion des erreurs}\label{gestion-des-erreurs}

Nous avons implémenté une gestion des erreurs afin de pouvoir afficher
des messages d'erreurs plus explicites.\\
Il est important de noter qu'on ne parle pas ici d'erreurs définies
explicitement par l'utilisateur, mais d'erreurs liées à une mauvaise
utilisation, à une détection par analyse statique ou à une erreur
d'évaluation.

\subsubsection{Analyse Statique}\label{analyse-statique-1}

En plus des garenties de typage, nous avons implémenté des erreurs de
typage afin de pouvoir afficher des messages d'erreurs plus explicites.

\begin{itemize}
\tightlist
\item
  Utilisation de variables non définies
\item
  Utilisation de constructeurs non définis
\item
  Double utilisation de variables au sein d'un même pattern
\end{itemize}

Lorsque l'une de ces erreurs est détectée, ou lorsque l'unification
échoue, nous affichons un message d'erreur explicite mentionnant la
ligne et la colonne de début et de fin de l'expression concernée, de
plus nous affichons l'équation qui a échoué.

\textbf{Exemples :}

\begin{Shaded}
\begin{Highlighting}[]
\KeywordTok{let}\NormalTok{ x = a }\KeywordTok{in}\NormalTok{ ()}
\end{Highlighting}
\end{Shaded}

Affichera :

\begin{Shaded}
\begin{Highlighting}[]
\ExtensionTok{Type}\NormalTok{ Inference:}
\ExtensionTok{Error}\NormalTok{ from line 1 col 8 to line 1 col 9: }
  \ExtensionTok{Unbound}\NormalTok{ variable a.}
\end{Highlighting}
\end{Shaded}

\begin{Shaded}
\begin{Highlighting}[]
\KeywordTok{let}\NormalTok{ x = (}\KeywordTok{fun}\NormalTok{ a {-}\textgreater{} a) }\KeywordTok{in}\NormalTok{ (x : }\DataTypeTok{int}\NormalTok{)}
\end{Highlighting}
\end{Shaded}

Affichera :

\begin{Shaded}
\begin{Highlighting}[]
\ExtensionTok{Type}\NormalTok{ Inference:}
\ExtensionTok{Error}\NormalTok{ from line 1 col 33 to line 1 col 36: }
  \ExtensionTok{int}\NormalTok{ = }\ErrorTok{(}\ExtensionTok{lambda}\NormalTok{ int,int}\KeywordTok{)}\BuiltInTok{.}
\end{Highlighting}
\end{Shaded}

\subsubsection{Erreurs d'évaluation}\label{erreurs-duxe9valuation}

Nous avons également implémenté des erreurs d'évaluation afin de pouvoir
afficher des messages d'erreurs plus explicites.\\

\begin{itemize}
\tightlist
\item
  Fuite dans les cas de filtrage
\item
  Division par zéro
\end{itemize}

Comme pour les erreurs de typage, lorsqu'une erreur d'évaluation est
détectée, nous affichons un message d'erreur explicite mentionnant la
ligne et la colonne de début et de fin de l'expression concernée.

\textbf{Exemples :}

\begin{Shaded}
\begin{Highlighting}[]
\NormalTok{(}\DecValTok{1}\NormalTok{ / ((}\KeywordTok{fun}\NormalTok{ a {-}\textgreater{} }\DecValTok{0}\NormalTok{) () )  ) }
\end{Highlighting}
\end{Shaded}

Affichera :

\begin{Shaded}
\begin{Highlighting}[]
\ExtensionTok{Evaluation:}
\ExtensionTok{Error}\NormalTok{ from line 1 col 0 to line 1 col 26: }
  \ExtensionTok{Division}\NormalTok{ by zero : }\ErrorTok{(}\ExtensionTok{1}\NormalTok{ / }\ErrorTok{(}\KeywordTok{(}\ExtensionTok{fun}\NormalTok{ a }\AttributeTok{{-}}\OperatorTok{\textgreater{}}\NormalTok{ 0}\KeywordTok{)} \KeywordTok{()))}\BuiltInTok{.}
\end{Highlighting}
\end{Shaded}

\begin{Shaded}
\begin{Highlighting}[]
\NormalTok{(}\KeywordTok{match} \DecValTok{2} \KeywordTok{with} 
 \DecValTok{1}\NormalTok{ {-}\textgreater{} ()}
\NormalTok{)}
\end{Highlighting}
\end{Shaded}

Affichera :

\begin{Shaded}
\begin{Highlighting}[]
\ExtensionTok{Evaluation:}
\ExtensionTok{Error}\NormalTok{ from line 1 col 0 to line 1 col 22: }
  \ExtensionTok{Non}\NormalTok{ exhaustive pattern match on :}
\KeywordTok{(}\ExtensionTok{match}\NormalTok{ 2 with }
 \ExtensionTok{1} \AttributeTok{{-}}\OperatorTok{\textgreater{}} \ErrorTok{(}\KeywordTok{)}
\KeywordTok{)}
\end{Highlighting}
\end{Shaded}

\section{Améliorations possibles}\label{amuxe9liorations-possibles}

Dans cette partie, nous allons présenter les différentes améliorations
que nous aurions pu implémenter dans notre langage.

Toutes ces améliorations ont été pensées et spécifiées, mais par manque
de temps, nous n'avons pas pu les mettre en place.

\textbf{Erreurs Utilisateur:}

La possibilité de définir des erreurs et d'en générer au sein du
programme aurait pu être une amélioration intéressante.

\textbf{Egalité structurelle :}

Nous aurions pu implémenter l'égalité structurelle afin de pouvoir
comparer des valeurs de type complexe comme les constructions.

\textbf{Analyse Statique :}

Voici les différentes améliorations que nous aurions pu implémenter dans
l'analyse statique :

\begin{enumerate}
      \item
            Exhaustivité du filtrage de motif
            Vérifier que le filtrage de motif était exhaustif. C'est-à-dire que toutes les valeurs possibles de l’expression filtrée sont couvertes par les patterns du filtrage de motifs.
      \item
            Détection de pattern inatteignable
            Detecter que le filtrage de motif ne contient pas de pattern inatteignable.\
            C'est-à-dire qu'un pattern n'est pas atteignable, car il est précédé d'un pattern qui couvre toutes les valeurs possibles de l'expression filtrée.
      \item
            Détection des variables non utilisées
            Lors de l'analyse statique, nous aurions pu vérifier que toutes les variables de motifs sont utilisés dans l'expression de retour de chaque branche.
      \item
            Détection des types non utilisés
            Vérifier que tous les types définis par l'utilisateur sont utilisés dans le programme.
\end{enumerate}


\end{document}
